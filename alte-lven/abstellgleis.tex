

%------------------------------------------------------------------------------
\begin{frame}[allowframebreaks]{DBMS}
  \metroset{block=fill}
\begin{alertblock}{Was ist ein Datenbankmanagementsystem (DBMS)?}
Ein Datenbankmanagementsystem ist eine Software zum Betrieb von
Datenbanksystemen, also zur Verwaltung von Datenbasen
\end{alertblock}

\begin{exampleblock}{Basisfunktionalität}
\begin{enumerate}
    \item Definition von Datenstrukturen (via Data Definition Language)
    \item Manipulation von Daten (Einfügen, Ändern, Löschen) (via Data Manipulation Language)
    \item Abfrage  eines Datenausschnitts (via Query Language)
    \item Dauerhafte Speicherung der Daten (nicht zwingend bei jeder DB)
\end{enumerate}
\end{exampleblock}

Erweiterte Funktionalität
Datenschutz (Zugriff auf Daten einschränken)
Chris Date: Security means protecting the data against unauthorized users“
”
Sicherstellung der Datenintegrität: Konsistenz durch Regeln
(korrekte und widerspruchsfreie Daten)
Chris Date: Integrity means protecting the data against authorized users“
”
Datensicherung / Recovery (z.B. bei Systemabsturz)
Synchronisation (Mehrere User, verteilte DBMS)

Vorteile von Datenbanksystemen
%. . . oder: Warum sollte man sich seinen Persistenz-Layer nicht selbst programmieren?
\begin{enumerate}
    \item Standards bzgl. der Datenorganisation (ANSI-SPARC-Modell)
    \item Datenunabhängigkeit
    \item \textbf{Abstraktion:} Ebenen (physisch/logisch/Benutzersichten)
    \item Transparenz zwischen Ebenen: definierte Schnittstellen
    \item Flexibilität: Trennung von Speicherung und Verwaltung
    \item Effizienter Datenzugriff (Speichern, Auslesen, Sortieren, \dots)
    \item Hohe Verfügbarkeit bei gleichzeitiger Sicherheit: viele (gleichzeitige) Zugriffe
    \item Redundanzvermeidung
\end{enumerate}
\end{frame}


%------------------------------------------------------------------------------
\begin{frame}[allowframebreaks]{Besonderheiten von Relationalen Datenbanken}
In Relationalen Datenbanken werden Mengen von Entitäten bzw. Beziehungen durch Relationen repräsentiert
Relation (Mathematik)
Menge von n-Tupeln
Tupel
Zusammenfassung von „Objekten“ in einer Liste
Reihenfolge ist von Bedeutung
Jedes Tupel repräsentiert eine Entität bzw. Beziehung, jede Position im Tupel ein Attribut
Z.B. (Müller , Franz , 12344837)

* Relationen repräsentieren stets Mengen von Entitäten bzw. Beziehungen
mehrere Entitäten desselben Typs
mehrere Beziehungen desselben Typs
* Relationale Datenbanken modellieren die Welt in dieser bestimmten, relationalen Perspektive
Relationale Datenbanken sind ein Metamodell
Relationale Datenbanken „zwingen“ uns, unsere Originale relational zu denken
Relationale Datenbanken realisieren einen eingeschränkten Blick auf die Welt
Erfüllen Relationale Datenbanken den Zweck von Modellen, das Original „besser als bisher in den Griff [zu bekommen]?“ (Stachowaik: Allgemeine Modelltheorie, 1973, S. 140)


Operationen mit Relationen (Tabellen)

Relationen (Tabellen) sind die Bausteine Relationaler Datenbanken
Attribute repräsentieren für den Modellbildner relevante Eigenschaften der Entitäten und Beziehungen
Über die Attribute lassen sich Entitäten bzw. Beziehungen vergleichen, sortieren und filtern
Tabelle/Datenbank: Zweckgebundenes Erfassungsschema

\end{frame}


%------------------------------------------------------------------------------
\begin{frame}[fragile]{Blocks}
  \metroset{block=fill}
  \begin{columns}[T,onlytextwidth]
    \column{0.4\textwidth}

    \column{0.6\textwidth}
\begin{sqlcode}
SELECT DISTINCT column_list
FROM table_list
    JOIN table ON join_condition
WHERE row_filter
ORDER BY column
LIMIT count OFFSET offset
GROUP BY column
HAVING group_filter;
\end{sqlcode}

  \end{columns}
\end{frame}


%------------------------------------------------------------------------------
\begin{frame}[fragile]{Blocks}
    Datenbankabfrage: Select-Statement
\begin{sqlcode}
SELECT DISTINCT column_list
FROM table_list
	JOIN table ON join_condition
WHERE row_filter
ORDER BY column
LIMIT count OFFSET offset
GROUP BY column
HAVING group_filter;
\end{sqlcode}
\end{frame}



%------------------------------------------------------------------------------
\begin{sqlcode}
CREATE TABLE Persons (
    ID int NOT NULL PRIMARY KEY,
    LastName varchar(255) NOT NULL,
    FirstName varchar(255),
    Age int
);

CREATE TABLE Persons (
    ID int NOT NULL,
    LastName varchar(255) NOT NULL,
    FirstName varchar(255),
    Age int,
    CONSTRAINT PK_Person PRIMARY KEY (ID,LastName)
);
\end{sqlcode}

\begin{sqlcode}
CREATE TABLE Orders (
    OrderID int NOT NULL PRIMARY KEY,
    OrderNumber int NOT NULL,
    PersonID int FOREIGN KEY REFERENCES Persons(PersonID)
);
\end{sqlcode}

%------------------------------------------------------------------------------



% Messy Data, Open Refine
Thaller Preußenproblem, überlieferte Texte = Kanon oder zufällig überlebt → repräsentativ für ihre Zeit??

Sprachliche Unschärfe
Geisteswissenschaften haben viel mit Sprache zu tun
Sprache ist genuin unscharf, mehrdeutig, redundant
Die Abbildung von schwammiger Sprache auf konkrete Daten ist daher schwierig

%TODO Joins


\begin{sqlcode}
    INSERT INTO durch_strasse_verbunden (see, hotel, distanz) 
    VALUES (1,15,156);
\end{sqlcode}



\section{How-To Slides}

      \begin{block}{OpenRefine}
      \begin{quote}
        OpenRefine is a powerful, free and open source tool that can be used for data cleaning.
        
        OpenRefine will automatically track any steps allowing you to backtrack as needed and providing a record of all work done
    \end{quote}
      \end{block}
      
      
      

\begin{frame}{Mehr Informationen}

ich hab hier ein hilfreiches Video gefunden, das beispielhaft ein ERM zu einer relationalen Datenbanken überführt:
\protect\url{https://www.youtube.com/watch?v=CZTkgMoqVss}


\href{mehr Information}{https://www.it-swarm.com.de/de/sql/deklarieren-sie-die-variable-sqlite-und-verwenden-sie-sie/939818168/}

\protect\url{it-swarm.com.deit-swarm.com.de}
sql — Deklarieren Sie die Variable in SQLite und verwenden Sie sie
Ich möchte eine Variable in SQLite deklarieren und in der Operation insert verwenden.
%Wie in MS SQL:declare @name as varchar(10) set name = 'name' select * from table where name = @name Zum Beispiel muss ich last_insert_row und benutze es in insert.I...


\protect\url{https://www.quora.com/Is-1-true-and-0-false-in-programming}
-> meist ist 0 true und nicht-0 false, aber zB in C ist eine Operation, die 0 zurückliefert erfolgreich verlaufen (success code) und bei nicht-0 mit Problemen (failure code)

    
\end{frame}







