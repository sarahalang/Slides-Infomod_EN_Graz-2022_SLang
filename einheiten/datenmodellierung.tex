

\section{Data modelling}


%------------------------------------------------------------------------------
\begin{frame}[allowframebreaks]{Data Modelling: Stachowiak's \emph{General Model Theory}}
\small 

\bg{alert}{white}{His notion of a model}
„Alle Erkenntnis ist Erkenntnis in Modellen oder durch Modelle und jede
menschliche Weltbegegnung überhaupt bedarf des Mediums Modell“ (Herbert Stachowiak 1973)
\alert{$\to$ All knowledge-making is knowledge-making in or through models and all human perception of the world needs models as a medium. }
\smallskip


\bg{alert}{white}{Data modelling}
Model = snippet of the real world but it only covers the attributes I chose to be relevant for the task at hand. 
Thus, the model and the aspect of the real world it models (its subject) diverge. 

\smallskip

\bg{alert}{white}{Model = abstraction (=class)}
concrete applications of the model (i.e. one concrete cooking recipe, poem, etc.) are called \emph{instances}.

\smallskip

Standardized models allow us to exchange and analyse data, search/query data. 
 $\to$ only \bg{alert}{white}{formal models} can be processed digitally, i.e. every digital model is a formal model.
\smallskip

\framebreak\small

\bg{alert}{white}{1) Mapping}
 {Models are always models of something, i.e. mappings from, representations of natural or artificial originals, that can be models themselves.}
\smallskip

\bg{alert}{white}{2) Reduction}
Models in general capture not all attributes of the original represented by them, but rather only those seeming relevant to their model creators and/ or model users. For example, I can only photograph (or even see!) a statue from one perspective at a time. I never see `the thing itself' but a partial aspect of all its defining properties. 

\smallskip

\bg{alert}{white}{3) Pragmatism}
 Models are not uniquely assigned to their originals per se. They fulfill their replacement function \\
 \footnotesize
    a) for particular – cognitive and/ or acting, model using subjects, \\
    b) within particular time intervals and \\
    c) restricted to particular mental or actual operations.

\smallskip


%\bg{alert}{white}{Modelle = vereinfachte Repräsentationen von Teilen der realen Welt.}

\end{frame}

%------------------------------------------------------------------------------
\begin{frame}[allowframebreaks]{McCarty}

\bg{alert}{white}{Willard McCarty, „Modeling: A Study in words and Meanings“ (2004)}
\begin{enumerate}
    \item A model is a representation of something, for example for research purposes.
    \item Modelling = hermeneutic process by which models are created and used to solve problems.
    \item \emph{model of} (descriptive) vs. \emph{model for} (i.e. architectural plan for building a house).
    \item modelling as an experimental process with the goal of capturing aspects of reality
\end{enumerate}
\framebreak

\bg{w3schools}{white}{Modelling as an interative process} \\
\begin{itemize}
    \item include present knowledge
    \item create model
    \item use for model to research the subject
    \item learn from discrepancies between model and real world to refine your model. 
    \item re-make the model to integrate new insights
    \item repeat
\end{itemize}

\end{frame}

%------------------------------------------------------------------------------
\begin{frame}[allowframebreaks]{Models = simplified repres. of parts of the real world }
\bg{w3schools}{white}{Model = mapping = representation $\neq$ original}: just parts of the original are represented. Choosing which parts depends on the intended use by which you later measure how useful (i.e. good) the model is. Or is the best model the most accurate? 

\bg{w3schools}{white}{subjective \& abstracted} $\to$ the model has to be created with reference to a specific research question and will only be useful (or the most useful) with regard to this question. Perception and representation are always selective \& subjective. Thus, so are the models. I can never capture all aspects at the same time. A photo of a vase only shows it from one side. 

\bg{w3schools}{white}{Metadata} can be used to enrich data models (supplying additional information, such as administrative or technical) 

\bg{w3schools}{white}{Disambiguation} $\to$ often necessary to represent something in the digital sphere (practical aspect of modelling), for example by using normalization, geonames, GND etc. $\to$ conventions so that we can be sure we are referring to the same thing on the internet.
\medskip

Where does the realistic `mapping' of reality end? At which point are we left with only a subjective perspective of reality ?

\framebreak

\bg{w3schools}{white}{What do we even mean by original?} The book whose materiality gets lost in the digital realm? The idea behind the book?
\smallskip

\bg{alert}{white}{Example Photogrammetry / Structure from Motion}

 (= creating 3D models from overlapping photos): \\
 \textbf{reality} $\to$ more info than on the images. \\
 \textbf{3D model:} you need 8 Pixels (the same pixel on 8 different images) for one point in the model $\to$ less information than the sum of the (at least 50) images used to create the model from. 
\end{frame}
%------------------------------------------------------------------------------

%------------------------------------------------------------------------------
\begin{frame}[standout]
    \alert{Readings} -- summary: \\
    McCarty, Modelling \\[1em]
    {\footnotesize Please \alert{\href{https://docs.google.com/presentation/d/1YrEMpuguN12pY02jj_RKidxV9G88WTt5cCtUQldCe8w/edit?usp=sharing}{create a 1 slide summary for your part per group}}; designate 1 person to present to the group. \\
    }
\end{frame}
