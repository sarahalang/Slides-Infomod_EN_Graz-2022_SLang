\section{Syllabus}
%------------------------------------------------------------------------------
\begin{frame}{Content}
Theoretical and practical introduction to modelling data in tabular formats. 

\begin{enumerate}
    \item tabular data
    \item relational model (Entity-Relationship-Model) 
    \item SQL databases
\end{enumerate}

\end{frame}
 
\section{Preliminaries}
%------------------------------------------------------------------------------
\begin{frame}{How to get a positive grade on this class}
\subsection{Grading}

  \begin{columns}[T,onlytextwidth]
    \column{0.48\textwidth}
      \begin{exampleblock}{Final Submission (60\%)}
\begin{itemize}\footnotesize
\item \textbf{Infomod:} Text, ER model, SQL database
\item \textbf{DigEd:} Small digital edition or review of an existing edition.
\item You can start in the last month of the semester and ask questions.
\item You can collaborate but no plagiarism (Uni Graz zero tolerance policy).
\end{itemize}
\end{exampleblock}

\begin{exampleblock}{Homework assignments (40\%)}\footnotesize
Communicated and to be completed within the week
\end{exampleblock}

    \column{0.48\textwidth}
      \begin{alertblock}{Other aspects}
\begin{enumerate}\scriptsize
    \item attendance in class (you can miss max. 3, to be communicated beforehand).
    \item Positive grade: at least 50\%  on all partial submissions.
    \item ``LVen mit immanentem Prüfungscharakter'' $\to$ once you accept the first task you get a grade (i.e. first homework this week)
    \item If you get a negative grade, the whole class needs to be retaken.
\end{enumerate}

\begin{quote}\scriptsize
    Nichterbringung weiterer Teilleistungen ohne wichtigen Grund ist Prüfungsabbruch (Negativbeurteilung). Abmeldung nach bereits übernommener Teilleistung führt zu negativer Beurteilung.
\end{quote}
\end{alertblock}

\end{columns}

\small 
see also: slides on grading \& further info materials on the final submission
    
\end{frame}


%------------------------------------------------------------------------------
\begin{frame}{Deadlines}

\begin{alertblock}{Hard deadlines}\small
\textbf{All deadlines are hard deadlines.} 
You can get extensions for good reasons.
\begin{itemize}
\item Good reasons for example: care responsibility, being ill, etc.
\item i.e. understandable reasons which are communicated asap
\end{itemize}
\end{alertblock}

\begin{alertblock}{If you miss a deadline\dots}
\begin{itemize}\small
\item If you didn’t communicate: negative grade.
\item Otherwise up for discussion according to the circumstances.
\end{itemize}
\end{alertblock}

\end{frame}
%------------------------------------------------------------------------------
\begin{frame}{Learning Goals}
\subsection{Learning Goals}
\begin{enumerate}
    \item get to know modelling theory („Allgemeine Modelltheorie“ nach Herbert Stachowiak)
    \begin{itemize}
        \item try the Entity-Relationship-Diagramm (Peter Chen)
        \item go from conceptual model to computer processable model
        \item represent relational data structures
    \end{itemize}
    \item learn SQL (\emph{Structured Query Language})
    \item get to know different data structures 
    \item basic practical modelling experience
\end{enumerate}

\metroset{block=fill}
\begin{alertblock}{Final project}
\footnotesize
Move from your own conceptual data model to a relational database in SQL and write about it (3--5 pages). 

\end{alertblock}
\end{frame}

%------------------------------------------------------------------------------
\begin{frame}{Literature on learning tech subjects}

\begin{enumerate}
    \item Carol Dweck, \emph{Mindset: The New Psychology of Success} (New York: Random House 2006). {\footnotesize (\href{https://www.youtube.com/watch?v=hiiEeMN7vbQ}{10min Video } | \href{https://latex-ninja.com/2021/06/21/the-most-important-book-to-read-if-you-want-to-learn-digital-humanities-computer-science-maths-programming-or-latex/}{blog post on the topic}) }
    \item K. Anders Ericsson \& Robert Pool, \emph{Peak: Secrets from the New Science of Expertise}, Penguin: London 2016. 
    \item Barbara Oakley, \emph{A Mind for Numbers}, NY 2014.
\end{enumerate}

\end{frame}
%------------------------------------------------------------------------------
%------------------------------------------------------------------------------
\begin{frame}{Working with computers as a humanities person}

\begin{itemize}
    \item \href{https://static.uni-graz.at/fileadmin/gewi-zentren/Informationsmodellierung/PDF/U__bungsblatt-0.pdf}{Übungsblatt 0} is a prerequisite
    \item don't panic
\end{itemize}
    
\end{frame}
%------------------------------------------------------------------------------

%------------------------------------------------------------------------------
\begin{frame}[standout]
    \alert{Present yourselves! } \\
    Name, pronouns, domain of origin, interests, etc.
\end{frame}


%------------------------------------------------------------------------------
%\begin{frame}[standout]    Resources\end{frame}
\subsection{Resources}
%------------------------------------------------------------------------------
\begin{frame}{References}
\begin{enumerate}
    \item \textbf{Stachowiak,} Herbert: \emph{Allgemeine Modelltheorie}, Wien 1973. 
    \item \textbf{Chen,} Peter: \emph{The Entity-Relationship Model -- Toward a Unified View of Data}, 1976.
    \item \textbf{Jannidis (Hg.):} \emph{Digital Humanities. Eine Einführung}, Stuttgart 2017. \\
    \begin{itemize}
        \item \href{https://link.springer.com/book/10.1007/978-3-476-05446-3}{Online-Exemplar} $\to$ gedrucktes Exemplar in der ZIM-Bibliothek.
        \item Kap. 7 (\emph{Datenmodellierung}) und 8 (\emph{Datenbanken}).
    \end{itemize}
    \item \textbf{Flanders, Julia; Jannidis, Fotis:} \emph{Knowledge Organization and Data Modeling in the Humanities}, 2015. 
    \item \textbf{Lothar Piepmeyer:} Grundkurs Datenbanksysteme. Von den Konzepten bis zur Anwendungsentwicklung, Hanser 2011. 
    \item \textbf{Ciula, Arianna \& Eide, Øyvind \& Marras, Cristina \& Sahle, Patrick (Ed.):} \emph{Models and Modelling between Digital and Humanities: A Multidisciplinary Perspective}, 2018.
    \item \textbf{Julia Flanders, Fotis Jannidis:} \emph{The Shape of Data in Digital Humanities Modeling Texts and Text-based Resources}, Routledge 2019.
\end{enumerate}

\end{frame}


%------------------------------------------------------------------------------
\begin{frame}{Ressources}

\begin{enumerate}
    \item \textbf{OpenRefine:}
    \begin{itemize}
        \item \href{https://openrefine.org/download.html\#openrefine-32}{OpenRefine Download}
        \item \href{https://openrefine.org/}{OpenRefine site with tutorials}
    \end{itemize}
    \item \textbf{SQL:}
    \begin{itemize}
        \item \href{https://sqlite.org/download.html}{SQLite Download}
        \item \href{https://sqlitebrowser.org/dl/}{SQLite Browser}
        \item \href{https://www.w3schools.com/sql/}{w3schools SQL Tutorial}
    \end{itemize}
    \item \textbf{ER:}
    \begin{itemize}
        \item \href{https://erdplus.com/standalone}{ERDplus Tool for creating ER models}
        \item \href{https://www.geeksforgeeks.org/introduction-of-er-model/}{GeeksforGeeks-Tutorial/Intro to ER Models}
    \end{itemize}
\end{enumerate}

\end{frame}
