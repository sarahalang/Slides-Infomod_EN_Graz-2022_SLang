
%----------------------------------


\begin{frame}[fragile,allowframebreaks]{XML-Regeln}
\begin{columns}
\footnotesize
\column{0.5\textwidth}
\bg{w3schools}{white}{Prolog}~ \\
\mycommand{<xml version="1.0" encoding="utf-8">}{XML-Deklaration}
\mycommand{<?xsl-stylesheet type="text/xsl" href="mein.xsl"?>}{Verarbeitungsanweisungen (optional)}

evtl. Einbindung des Dokumentmodells (optional) \\
DTD, XML Schema, RelaxNG, Schematron \\

\column{0.5\textwidth}
\bg{w3schools}{white}{Entitäten}~ `geschützte Zeichen', da sie eine Bedeutung in der Metasprache und der Objektsprache haben; selbstdefiniert \& vordefinierte, z.B.: \\
\mycommand{&lt;}{<}
\mycommand{&gt;}{>}
\mycommand{&amp;}{\&}
\end{columns}


\end{frame}





\section{Text Encoding Initiative}
%-----------------------------------------------------
\begin{frame}[fragile,allowframebreaks]{TEI Primer}
\footnotesize
\bg{alert}{white}{Text Encoding Initiative}\\
\bgupper{w3schools}{black}{.xml}\\
XML-Standard \item De-Facto-Standard in den \emph{Digital Humanities} \item Kodierung von Druckwerken (Editionswissenschaft) \item bishin zu Linguistik
\begin{quote}
    Die Text Encoding Initiative (TEI) ist eine 1987 gegründete Organisation (seit 2000 als TEI-Konsortium organisiert) und ein gleichnamiges Dokumentenformat zur Kodierung und zum Austausch von Texten, das diese entwickelt hat und weiterentwickelt.  (\href{https://de.wikipedia.org/wiki/Text_Encoding_Initiative}{Wiki})
\end{quote}

\begin{myxml}{TEI-Grundstruktur}
<TEI> <!-- Wurzelelement -->
    <teiHeader> ... </teiHeader> <!-- Autor, Titel, Datierung, Quelle(n), Editionsrichtlinien,
Versionierung, etc. -->
    <text> ... </text>
</TEI>
\end{myxml}
\smallskip

\bg{w3schools}{white}{Ressourcen}\\
\href{http://www.tei-c.org/Support/Learn/}{Learn TEI} \item \href{http://www.tei-c.org/support/learn/teach-yourself-tei/}{Teach Yourself} \item P5 = 5. Proposal \item MEI für Musik \item CEI für Charters (Urkunden) \item \href{http://www.tei-c.org/}{http://www.tei-c.org/} \item Standard zur Textkodierung \item Publikationswerkzeuge: Versioning Machine, TEI Boilerplate, TEICHI, TEI
Stylesheets 
\end{frame}

%-------------------------------------
\begin{frame}[fragile,allowframebreaks]{TEI Header}
\footnotesize
\bg{alert}{white}{fileDesc}~ = Instrumentarium zur umfassenden bibliographischen  Beschreibung der Inhalte eines TEI-Dokumentes  \\
\bg{alert}{white}{encodingDesc}~ = beschreibt den Zusammenhang des elektronischen Textes mit
dem Quelltext, z.B. diverse Regel bei der Transkription, sowie Erläuterungen zum
Annotationsprozess \\

\begin{myxml}{TEI-Header}
<TEI> <!-- Wurzelelement -->
    <teiHeader>
        <fileDesc> ... </fileDesc> <!-- obligatorisch --> 
        <encodingDesc> <!-- optional -->
        <profileDesc> <!-- optional -->
        <revisionDesc> <!-- optional -->
    </teiHeader> 
    <text> ... </text>
</TEI>
\end{myxml} 
\bg{alert}{white}{profileDesc}~ = Beschreibung aller nicht bibliographischen Aspekte des
Textes, z.B. Informationen über die Entstehung des Textes, sowie über verwendete Sprachen
u.Ä. \\
\bg{alert}{white}{revisionDesc}~ = Beschreibung aller Änderungen und Überarbeitungsschritte am
Transskript des Quelltextes.
\end{frame}
%----------------------------------
\begin{frame}[allowframebreaks]{TEI Basics}
\footnotesize
TEI = \bg{alert}{white}{modular} $\to$ im Schema können Untermengen festgelegt werden (`Ich benutze Core und \dots ') \item \href{http://www.tei-c.org/Roma/}{ROMA Schema} \item sonst TEI P5 All
\smallskip

\bg{w3schools}{white}{Gemeinsamkeiten von Textdokumenten: } \\
\bg{alert}{white}{\href{http://www.tei-c.org/release/doc/tei-p5-doc/en/html/CO.html}{TEI Core} }

\begin{itemize}\footnotesize
    \item Identifikatoren (Seitenangaben, Signatur, Inventarnummer, Regalnummer, etc.)
    \item Abschnitte und Unterabschnitte (Gliederung)
    \item Abbildungen, Skizzen, grafische Elemente
    \item Schreibmodus (Prosa, Drama, Vers etc.)
    \item Strukturelle Einheiten (Absatz, Listen, Strophen, Zeilen, Reden, etc.)
    \item Textunterschiede oftmals durch unterschiedliche Formatierung gekennzeichnet (Titel, Überschriften, Zitate, Betonungen, etc.)
    \item Sachinformationen (Personen, Orte, etc.)
    \item Texteingriffe Korrekturen, Streichungen, Revisionen
\end{itemize}
\vspace{1em}

\bg{w3schools}{white}{Warum TEI verwenden oder nicht?} \\
\textbf{Nachteil:} man muss Regeln befolgen -- muss man aber sonst sinnvollerweise eh auch. Wollte man es wirklich gut machen, wäre es viel Arbeit! Andererseits: Flexibilität. TEI hat nicht immer etwas für Spezialzwecke. \\
\textbf{Vorteil:} die Regeln hat sich schon jemand überlegt, man muss sie nur nachschauen, verstehen und anwenden. Allgemeiner Standard in den Geisteswissenschaften, macht Daten unterschiedlicher Projekte interoperabel.
\end{frame}


%-----------------------------------
\begin{frame}[fragile,allowframebreaks]{TEI verwenden}
\footnotesize
\href{http://www.tei-c.org/release/doc/tei-p5-doc/en/html/SG.html}{Gentle Intro to XML}

\bg{alert}{white}{TEI Core}~ \textbf{div} (Abschnitt) \item \textbf{p} (Paragraph) \item \textbf{head} (Überschrift) \item \textbf{lb} (linebreak) \item \textbf{pb} (page break / beginning) \item \textbf{hi} (highlight) \item \textbf{l} (line) \item \textbf{lg} (line group) \item \textbf{list} \item \textbf{item} \item \textbf{listBibl} \item \textbf{bibl} (bibliograph. Angabe)

\bg{alert}{white}{Attribute}~ \textbf{@n} (label) \item \textbf{@type} (Typisierung) \item \textbf{xml:id} (eindeutige Kennung) \item \textbf{xml:lang} (Sprache) \item \textbf{@rend} (Darstellung) \item \text{@ana} (Interpretation)

\begin{myxml}{Typische Anwendungen}
<foreign xml:lang="en">word</foreign>
<term type="homonym"/>
<date when="2009-04-27"/>
<time when="12:00:00"/>
<name type="person"/>
<persName n="Caesar" xml:id="#44BC">Caesaris</persName> <!-- oder -->
<persName key="ID.01.208"/>
<person/>
<emph/> <hi rend="italic">kursiver Text</hi>
<seg/> <abbr type="acronym"/>
<placeName xml:id="#Whitby">Abbey</placeName>
\end{myxml}

\bg{alert}{white}{Namensräume}~ Identifikation über URI \item \bg{w3schools}{white}{<präfix:name>}~ \item z.B. <tei:p> (`Ich meine das <p> nach dem TEI-Standard'). \\
\bg{alert}{white}{Deklaration}~ <element xmlns=“URI“> \dots \\
<präfix:element xmlns:präfix=“URI“> \dots \\
Z.B. <tei:p xmlns:tei=“http://www.tei-c.org/ns/1.0“>...


\end{frame}



\begin{frame}[fragile,allowframebreaks]{Spezialfälle: Rede \& Briefe}
\footnotesize
Kodierung von Sprechakten  (\href{http://www.tei-c.org/release/doc/tei-p5-doc/en/html/examples-sp.html}{TEI-Referenz}), falls Speaker davor steht, sonst \href{http://www.tei-c.org/release/doc/tei-p5-doc/en/html/examples-said.html}{TEIs `said'}:
\begin{verbatim}
<sp who="#person">
    <speaker>1.</speaker> <p>Bla, bla, bla.</p>
</sp>

<said who="#Adolphe">- Alors, Albert, quoi de neuf?</said>
\end{verbatim}

\begin{myxml}{Briefe in TEI \href{http://www.tei-c.org/release/doc/tei-p5-doc/en/html/DS.html#DSOC}{(Referenz)}}
<div type="letter" n="14">
    <head>Letter XIV: Miss Clarissa Harlowe to Miss Howe</head>
        <opener>
            <dateline>Thursday evening, March 2.</dateline>
            <salute>Hallo,</salute>
        </opener>
    <p>On Hannah's depositing my long letter ...</p>
    <closer>
        <salute>Yours more than my own,</salute>
        <signed>Clarissa Harlowe</signed>
    </closer>
</div>
\end{myxml}
\end{frame}

\begin{frame}[fragile,allowframebreaks]{Annotation fortgeschritten}
\footnotesize
\bg{w3schools}{white}{\emph{Named Entities} \& Indirekte Referenz}\\
\href{http://www.tei-c.org/release/doc/tei-p5-doc/en/html/ND.html}{TEI 13: Names, Dates, People, Places} \item 
\textbf{persName} für Namensnennung, \textbf{<rs>} für \emph{referring string} bei indirekter Nennung (`er', `der Herr', etc.). Hier dann mit \textbf{@key} oder \textbf{@ref} spezifizieren, wer gemeint ist. (\href{http://www.tei-c.org/release/doc/tei-p5-doc/en/html/CO.html#CONARS}{Referenz}). \textbf{forename} \item \textbf{surname} \seo \textbf{roleName} (z.B. `König') \item  \textbf{genName} (`der Ältere') \item \textbf{addName} \item \textbf{nameLink} (`von').

\begin{verbatim}
<name role="writer" type="person"
ref="http://d-nb.info/gnd/118540238">
Goethe</name>
<person>
  <addName type="Former">Murray</addName>
  <forename>Wilhelmina</forename>
  <addName type="nickname">Mina</addName>
</person
\end{verbatim}

Eine \textbf{person} (selbst) ist nicht identisch mit dem \textbf{persName} (Name)! Dafür zur Verfügung stehende Elemente: \textbf{persName} zu \textbf{person}, \textbf{orgName} für \textbf{org} (Organisation), \textbf{placeName} für \textbf{place.} \textbf{geogName} (= geographical) = Lanschaftsmarker, Berge, etc.
Problem: \textbf{Unterscheidliche Namensformen}, daher:
\bg{w3schools}{white}{Normalisierung:} Zeigen, dass \emph{dieselbe} Person gemeint ist. Wir können also eine sog. `normalisierte' Form angeben, z.B. in Form einer Referenz auf die Personenliste im TEI-Header, auf deren \textbf{@xml:id} wir per \textbf{@ref} verweisen. Oder Normdaten der GND oder in Attribut (z.B. \textbf{@n} (\emph{label}) oder \textbf{@ana} (Interpretation)). 

\bg{w3schools}{white}{Redundanz ist eine Fehlerquelle} $\to$ Referenz auf eindeutigen Ort, wo die Info genau 1x vorhanden ist $\to$ Bei Fehlern nur 1x ausgebessern, nicht 200 Okkurenzen. Zur eindeutigen Referenzierung xml:ids, z.B. \verb|<person xml:id="Mina">Mina</person>|, darauf refernzieren dann mithilfe des \texttt{@ref}-Attributs und einem Hashtag: \verb|<persName ref="#Mina">Mina</persName>|.

\end{frame}
