


\section{Messy Data}
%TODO Open Refine (Slides in Moodle und Ordner)


%------------------------------------------------------------------------------
\begin{frame}[standout]
    \alert{Lektüre}-Zusammenfassung: \\
    Christof Schöch, \emph{Big? Smart? Clean? Messy? Data in the Humanities} \\[1em]
    {\footnotesize Bitte \alert{\href{https://docs.google.com/presentation/d/1JweC-x1MxLjoKcDcHDtsw6NoYZfSbQ0USVYzYxqVx_c/edit?usp=sharing}{hier im Google Slides zusammenfassen}}; 1 Person stellt dann vor. \\
    }
\end{frame}


%------------------------------------------------------------------------------
\begin{frame}[allowframebreaks]{Das Thallersche Preußenproblem}
\metroset{block=fill}

\alert{Historische Daten: Das Thallersche Preußenbeispiel} \\
{\scriptsize
Das Preußenbeispiel findet sich in: Manfred Thaller, \emph{Gibt es eine fachspezifische
Datenverarbeitung in den historischen Wissenschaften?}, in: K.H. Kaufhold und J.
Schneider, \emph{Geschichtswissenschaft und elektronische Datenverarbeitung},
Wiesbaden 1988, S. 45--83, hier S. 57--66.

}

\begin{columns}
  \column{0.45\textwidth}
  \begin{exampleblock}{Die Aufgabe}\footnotesize
  Suche Personen, die aus Preußen stammen (d.h. in Preußen geboren wurden), jünger als 50 Jahre sind \& ein Vermögen von mehr als 100 Einheiten der Währung XY besitzen
  \end{exampleblock}
  \column{0.55\textwidth}
  \begin{alertblock}{Problem 1: Preußen}\scriptsize
  \begin{itemize}
      \item Kommt Preußen überhaupt in den Daten vor -- oder gibt es nur einen Geburtsort? Idealerweise sollte das System den Geburtsort in Preußen lokalisieren (z.B. via GIS).
      \item Preußen ist ein historisches Gebilde, das mehrfach seine Grenzen verändert hat. Man braucht ein möglichst genaues Datum um die Grenzen festzulegen.
      \item Aber: Datum ist relativ zum Alter (und dieses wiederum relativ zum Entstehungsdatum der Quelle).
  \end{itemize}
  \end{alertblock}
\end{columns}

{\scriptsize\flushright Quelle: LV Datenbanken, Gunter Vasold, SS17

}

\framebreak 

\begin{columns}
  \column{0.45\textwidth}
  \begin{exampleblock}{Die Aufgabe}\footnotesize
  Suche Personen, die aus Preußen stammen (d.h. in Preußen geboren wurden), jünger als 50 Jahre sind \& ein Vermögen von mehr als 100 Einheiten der Währung XY besitzen
  \end{exampleblock}
  \column{0.55\textwidth}
  \begin{alertblock}{Problem 2: Das Alter}\scriptsize
  \begin{itemize}
      \item Altersangaben in historischen Quellen sind in der Regel Schätzungen.
      \item Das Alter muss relativ zur Entstehungszeit der Quelle ermittelt werden.
      \item Dazu muss das Kalendersystem und Kalenderstil bekannt sein (zeitl. und räumlich unterschiedlich).
      \item Die Datenbank sollte mit unscharfen Angaben umgehen können: ``ca. 50 oder jünger''. 
      \item Idealerweise sollte das System den Grad der Unschärfe selbständig ermitteln (z.B. statistisch) und anwenden können.
      \item Daraus sollte bei Bedarf auch eine Gewichtung zu errechnen sein.
  \end{itemize}
  \end{alertblock}
\end{columns}
  
{\scriptsize\flushright Quelle: LV Datenbanken, Gunter Vasold, SS17

}

\framebreak 

\begin{columns}
  \column{0.45\textwidth}
  \begin{exampleblock}{Die Aufgabe}\footnotesize
  Suche Personen, die aus Preußen stammen (d.h. in Preußen geboren wurden), jünger als 50 Jahre sind \& ein Vermögen von mehr als 100 Einheiten der Währung XY besitzen
  \end{exampleblock}
  \column{0.55\textwidth}
  \begin{alertblock}{Problem 3: Das Vermögen}\scriptsize
  \begin{itemize}
      \item Historische Währungen umrechnen auf neutrale und vergleichbare Skala.
      \item \emph{Voraussetzung:} Zeitpunkt (Datum) ermitteln (siehe oben).
      \item \emph{Voraussetzung:} Für dieses Datum und den Ort feststellen, in welchem politischen Gebilde (Territorium) sich der Ort befindet.
      \item Umrechnungsfaktoren für Währung für die Zeit und das Territorium errechnen.
      \item Damit Vergleichssumme errechnen und vergleichen.
  \end{itemize}
  \end{alertblock}
\end{columns}

{\scriptsize\flushright Quelle: LV Datenbanken, Gunter Vasold, SS17

}

\end{frame}



%------------------------------------------------------------------------------
\begin{frame}[standout]
    \alert{Lektüre}-Zusammenfassung: \\
    Michael Piotrowski, \emph{Serial Sources} \\[1em]
    {\footnotesize Bitte \alert{\href{https://docs.google.com/presentation/d/1JweC-x1MxLjoKcDcHDtsw6NoYZfSbQ0USVYzYxqVx_c/edit?usp=sharing}{hier im Google Slides zusammenfassen}}; 1 Person stellt dann vor. \\
    }
\end{frame}


