\section{Grundlagen der Informationsmodellierung, (Datenmodellierung)}


\begin{tabular}{l|l}
Titel & Grundlagen der Informationsmodellierung \\
Untertitel & Grundlagen der Datenmodellierung \\
Nummer & 562.021 \\
Art & Vorlesung und Übung \\
Semesterstunden & 2 \\
Angeboten im Semester & Sommersemester 2021 \\ \hline 
Ziel & Die Teilnehmer:innen erstellen auf Basis des Erlernten (s. Inhalt) ein eigenes Datenmodell und reflektieren dieses. \\
Inhaltliche Voraussetzungen & Grundlegende Computerkenntnisse \\
Lehr- und Lernmethode & interaktiv mit Neuen Medien (Wissensvermittlung zusätzlich computerbasiert, z.B. eLearning) $\to$ komplett online, aber \textbf{nicht} `asynchron', d.h. die LV findet größtenteils im vorgegebenen Zeitraum statt.\footnote{Dies liegt z.B. daran, dass bei praktischen Übungen dann sofort nachgefragt werden kann und Gruppenarbeit möglich ist.} \newline 
Theoretische Einführung in das Thema. \newline 
Selbständiges Erarbeiten und Festigen des Gelernten durch praktische Übungen. \newline 
Online Audio- und Videoübertragung \\
Beurteilung & Mitarbeit, Hausübungen, Abschlussprojekt \\
Anmerkung & Für den Kurs wird ein Laptop benötigt. \\
\textbf{Termin} & Mo, 13:30--15:00 (beginnend 08.03.2021)
\end{tabular}

\section{Inhalt}
Die Lehrveranstaltung führt Sie in grundlegende Aspekte der Modelltheorie und in Techniken der Datenmodellierung ein. Zu Beginn der Lehrveranstaltung werden Ihnen allgemeine Merkmale und Funktionen von Modellen vermittelt, wie sie in der „Allgemeinen Modelltheorie“ nach Herbert Stachowiak beschrieben sind. Dabei steht das Modell als Instrument der Erkenntnisgewinnung im Mittelpunkt. Danach werden Besonderheiten und wichtige Begriffe der Datenmodellierung erläutert. Schließlich wird Ihnen ein Weg aufgezeigt, wie Sie von der Konzeption zu einem fertigen Datenmodell gelangen. Dies umfasst im Einzelnen: das „Entity-Relationship-Diagramm“ nach Peter Chen, um das Konzept Ihres Modells zu visualisieren; die Überführung des konzeptionellen Modells in eine logische Struktur, die maschinell prozessiert werden kann; und schließlich die Repräsentation des Modells mithilfe des Computers. Es werden in der Lehrveranstaltung unterschiedliche Datenstrukturen vorgestellt (relational; hierarchisch; netzwerkartig), der Fokus liegt jedoch auf relationalen Datenmodellen.



\section{Literatur}

Stachowiak, Herbert: Allgemeine Modelltheorie, Wien 1973.
Jannidis (Hg.): Digital Humanities. Eine Einführung, Stuttgart 2017.
Peter Chen: The Entity-Relationship Model – Toward a Unified View of Data, 1976.
Ciula, Arianna (Ed.); Eide, Øyvind (Ed.); Marras, Cristina (Ed.); Sahle, Patrick (Ed.): Models and Modelling between Digital and Humanities: A Multidisciplinary Perspective, 2018.
\newpage 

Inhalt der einzelnen Einheiten
Einheit 1: Einführung





Vorstellung
Administratives
Modellbegriff
\item Stachowiak
\item McCarty
Modellierung als Erkenntnisprozess
\item Praktische Übung zur Modellierung von „Gegenständen“
\item Praktische Modellierungsbeispiele
Ausblick auf Lehrveranstaltungsziele anhand von ausgewählten Beispielen
Einheit 2: Alltäglicher Umgang mit Datenmodellen

Datenmodellierung im alltäglichen Computergebrauch
\item Textverarbeitungsprogramme

Formatvorlagen
Literaturverwaltung

Vergleich von Literaturverwaltungsprogrammen

Exportmöglichkeiten (BibTeX)

Datenaustausch
Einheit 3: Tabellen als Modellierungswerkzeug I

Tabellenkalkulation
\item Filter, Sortierung
\item Normalisierung
\item Formeln, Bedingte Formatierung
Einheit 4: Tabellen als Modellierungswerkzeug II


Praktisches Beispiel
\item Dariah-Geobrowser
\item Google Maps
\item Geonames
Speichern als „XML“
Einheit 5: Markup Languages I





Grundlagen von XML
Baumstruktur
Syntax
Elemente, Attribute
Beispiel
Enheit 6: Markup Languages II
 HTML
 KML
 SVG
Einheit 7: Markup Languages III


TEI
LIDO
Einheit 8: Schemasprachen




Wozu Schema?
Bildung von Semantik
Überblick über Schemasprachen: DTD, XML Schema, Relax NG
Grundlagen der DTDEinheit 9: Abfragesprachen: XPath I



Achsen
Bedingungen
Funktionen
Einheit 10: Entity-Relationship-Modell




Grundlagen
Komponenten
Beziehungen
\item Generalisierung/Spezialisierung (is-a)
\item Aggregation (is-part-of)
Kardinalitäten
Einheit 11: Entity-Relationship-Modell

Notationen
\item Chen-Notation
\item Bachman-Notation
\item UML
Einheit 12: Entity-Relationship-Modell

Beispielimplementation
Einheit 13: Zusammenfassung der Modell
 Überblick über unterschiedliche Modelltypen

o Tabellarische Modelle
o Hierarchische Modelle
o Relationale Modelle
Vergleich
Einheit 14: Abschluss
 Klausur











\newpage 
\subsection{Inhalte}
    \item Stachowiak „Allgemeine Modelltheorie“ als theoretischer Unterbau
    \item ERM als Grundlage der (relationalen) Datenmodellierung
    \item Jannidis: Datenmodellierung (Matrix)
    \item Relationales Datenmodell
    \item Alternative Datenmodelle: OHCR/XML, JSON (objektorientiert?)
    \item Aussicht auf weitere Datenmodelle/Datenmodellierung II (Graphen/RDF; UML?)

\subsection{Anwendungen}
    \begin{itemize}
        \item  (Daten)modellierung des Originals „Erde“
        \begin{itemize}
        \item Analog: Globus (Beispiel)
        \item Relationale Datenbank/Relationen  SQL
        \item CSV/Tabelle  Dariah Geobrowser 
        \item XML/OHCO  Dariah Geobrowser; Google Earth
        \item JSON?
        \end{itemize}
    \end{itemize}

\subsection{Übungen}
\begin{itemize}
    \item CSV $\to$ Texteditor
    \item XLSX $\to$ Excel $\to$ Tabellenoperationen
    \item OpenRefine
    \item SQL $\to$ Abfragen/Operationen
    \item Wie würdet ihr einen Sessel im Seminarraum identifizieren? Wie beschreiben?
\end{itemize}


\section{Abschlussprojekt/Klausur}

%Was bietet sich an?

%Mögliches Abschlussprojekt: Studenten entwerfen eigene ERM eigener Originale
%Studenten überführen ERM in ein physisches Datenmodell mithilfe von SQL (oder XML, JSON?)